\section{Parameterized Hybrid MDPs}
\label{sec:hybrid_mdps}

In this section we introduce Parameterized Hybrid Markov Decision Processes (PHMDPs) and show how the framework can be specialized to address multi-objective reward criteria, investigate parameter sensitivity and optimize continuous non-convex policy parameters. We also review their finite horizon solution via dynamic programming.

\subsection{Definition}
\label{sec:phmdp_def}

A parameterized hybrid Markov Decision Process (PHMDP) is defined by the tuple {\footnotesize \PMDPTuple}. {\footnotesize \State} specifies
a vector of states given by {\footnotesize $(\vec{d}, \vec{x}) =  \left( d_1, \ldots, d_m, x_1, \ldots, x_n \right) $}, where each {\footnotesize $ d_i \in \left\lbrace 0, 1 \right\rbrace $} {\footnotesize $\left( 1 \leq i \leq m \right)$} is discrete and each
{\footnotesize$ x_j \in \Real $} {\footnotesize $\left( 1 \leq j \leq   n \right)$} is continuous. {\footnotesize \Action} specifies a
finite set of actions.  {\footnotesize $\vec{\theta} \in \Theta$} are free parameters from the parameter space {\footnotesize $ \Theta $}. PHMDPs are naturally factored~\parencite{Boutilier_JAIR_1999} in terms of the state variables {\footnotesize$\vec{d}$} and {\footnotesize
$\vec{x}$}. Hence, the joint transition model can be written as:
{\footnotesize
\abovedisplayskip=0pt
\belowdisplayskip=0pt
\begin{align}
    \label{eq:hmdp_tfunc}
    \Transition :& \, \ProbArg{\vec{d}', \vec{x}' | \vec{d}, \vec{x}, a, \vec{\theta}} = \nonumber \\
    & \prod_{i=1}^{m} \ProbArg{d_i' | \vec{d}, \vec{x}, a, \vec{\theta}} \prod_{j=1}^{n} \ProbArg{x_j' | \vec{d}, \vec{d}', \vec{x}, a, \vec{\theta}}.
\end{align}   
}

{\footnotesize \RewardFunc} is the reward function which encodes the preferences of the agent. {\footnotesize \Horizon} represents the
number of decision steps until termination and the discount factor {\footnotesize $\gamma \in [0, 1)$} is used to geometrically discount
future rewards.

A policy {\footnotesize $\pi : \State \rightarrow \Action$}, specifies
the action to take in every state. The value of a state {\footnotesize
  $(\vec{d}, \vec{x}) \in \State$} under a given policy
{\footnotesize$\pi$} is given by: {\footnotesize
    \abovedisplayskip=0pt
    \belowdisplayskip=0pt
\begin{align*}
    V^{\pi}\left( \vec{d}, \vec{x}; \vec{\theta} \right) &= \Exp{\sum\limits_{h = 0}^{\Horizon} \gamma^h \cdot r^h | \vec{d}, \vec{x}, \vec{\theta}} \nonumber  
%    &= \sum\limits_{s' \in \State}  \Transition\left(s, \pi\left(s\right), s' \right) \left[ \Reward\left(s, \pi(s), s'\right) + \gamma \cdot V^{\pi}(s')\right]. \label{eq:mdp_bellman_eq}
\end{align*}
}
%Equation~\eqref{eq:mdp_bellman_eq} is known as the Bellman equation for $V^{\pi}$.

The state-action value function {\footnotesize$Q : \State \times \Action \rightarrow \mathbb{R}$} is given by:
{\footnotesize 
    \abovedisplayskip=0pt
    \belowdisplayskip=0pt
\begin{align}
    \label{eq:mdp_qfunc}
    Q^{\pi}\left(\vec{d}, \vec{x}, a; \vec{\theta}\right) = \Exp{\sum\limits_{h = 0}^{\Horizon} \gamma^h \cdot r^h | \vec{d}, \vec{x}, a, \vec{\theta}}. 
\end{align}
}

The value function of the optimal policy {\footnotesize$ \pi^{*} $} satisfies:

{\footnotesize 
\abovedisplayskip=0pt
\belowdisplayskip=0pt
\begin{align}
    \label{eq:opt_vfunc}
    V^{\pi^{*}}(\vec{d}, \vec{x}; \vec{\theta}) &= \max_{a \in A} \left\{ Q^{\pi^{*}}(\vec{d}, \vec{x}, a; \vec{\theta}) \right\}. 
\end{align}
}%

We again remark that in our formulation of PHMDPs the
parameters {\footnotesize $ \vec{\theta} $} are free parameters and not
learned from reward and transition samples as discussed in Related Work.
We also remark that we are not assuming a (Bayesian) reinforcement learning setting,
but rather a setting of strict uncertainty over {\footnotesize $ \vec{\theta} $} and
a parameterized symbolic value iteration solution to solve it.

In the subsequent sections we demonstrate how the PHMDP framework can
be specialised into models capable of: (i) investigating
multi-objective reward criteria; (ii) exact parameter sensitivity
analysis; and (iii) optimization of continuous non-convex policy
parameters.

\subsubsection{Multi-objective PHMDPs}

The PHMDP framework can be specialised into a multi-objective setting
by specifying the reward as {\footnotesize \MORewardFunc} where
{\footnotesize $ d $} is the dimension of the reward vector. In this
formulation the parameter {\footnotesize $ \vec{\theta}^{d} $} specifies the
linear scalarization over the reward components (i.e, {\footnotesize $\langle
\Reward, \vec{\theta}^{d} \rangle$}) and the value function can be expressed
as a function of all possible scalarizations.

\subsubsection{Sensitivity Analysis for PHMDPs}

In sensitivity analysis, we assume that {\footnotesize $\vec{\theta}^{d}$} are unknown
parameters and we solve directly for the value function in
Equation~\eqref{eq:opt_vfunc} as a parametric function
{\footnotesize $V^{\pi^{*}}(\vec{d}, \vec{x}; \vec{\theta}^{d})$} of these parameters.  This
parametric value function can then be directly analyzed or displayed in an
(Interactive) Decision Map, an example of which is shown in
section~\ref{sec:results_influenza}, which allows a user to explore
trade-offs between parameters {\footnotesize $\vec{\theta}^{d}$ }for the optimal policy as a
function of other state variables {\footnotesize $\vec{d},\vec{x}$}.

\subsubsection{Policy Gradient Methods for PHMDPs}

PHMDP policy parameters can be analyzed and optimised by parameterizing a policy {\footnotesize $\pi$} by action parameters {\footnotesize $\vec{\theta}^{d}$}, and evaluating the value w.r.t. this parametric policy, an example of which will be shown in section~\ref{sec:results_oe}. The resulting (non-convex) policy value as a function of {\footnotesize $\vec{\theta}^{d}$} can be
analyzed, and optimised using the parameteric form of {\footnotesize $V^{\pi^{*}}(\vec{d}, \vec{x}; \vec{\theta}^{d})$} and all symbolic
derivatives (up to any order) of this value function.

\subsection{Solution Methods}

Value iteration (VI)~\parencite{Bellman_PU_1957} is a general dynamic programming algorithm used to solve Markov Decision Processes (MDPs). In this section we modify VI to solve PHMDPs. The key idea of VI is to successively approximate {\footnotesize $V^{\pi^{*}}(\vec{d}, \vec{x}; \vec{\theta})$} and {\footnotesize $Q^{\pi^{*}}(\vec{d}, \vec{x}, a; \vec{\theta})$} by {\footnotesize $V^{h}(\vec{d}, \vec{x}; \vec{\theta})$} and
{\footnotesize $Q^{h}(\vec{d}, \vec{x}, a; \vec{\theta})$}, respectively, at each horizon {\footnotesize$h \in \Horizon$}. These two functions satisfy the following recursive relationship:
{\footnotesize 
    \abovedisplayskip=0pt
    \belowdisplayskip=0pt
    \begin{align}
        Q^{h}(\vec{d}, \vec{x}, a; \vec{\theta}) = \Reward(\vec{d}, \vec{x}, a; \vec{\theta}) + \gamma \, \cdot &  \nonumber \\ 
        \sum_{\vec{d}'} \int_{\vec{x}'} \ProbArg{\vec{d}', \vec{x}' | \vec{d}, \vec{x}, a; \vec{\theta}} & \cdot V^{h-1}(\vec{d}, \vec{x}; \vec{\theta}) \, d\vec{x}'  \label{eq:vi_qfunc} \\
        V^{h}(\vec{d}, \vec{x}; \vec{\theta}) = \max_{a \in A} \left\{ Q^{h}(\vec{d}, \vec{x}, a; \vec{\theta}) \right\} & \label{eq:vi_vfunc}
    \end{align}
}%

{\footnotesize $\ProbArg{\vec{d}', \vec{x}' | \vec{d}, \vec{x}, a; \vec{\theta}}$ } is specified in Equation~\eqref{eq:hmdp_tfunc}. The algorithm can be executed by first initialising {\footnotesize   $V^{0}(\vec{d}, \vec{x}; \vec{\theta})$} to zero or the terminal reward. Then for each {\footnotesize$h$}, {\footnotesize $V^{h}(\vec{d}, \vec{x}; \vec{\theta})$} is calculated from {\footnotesize $V^{h-1}(\vec{d}, \vec{x}; \vec{\theta})$} via Equations~\eqref{eq:vi_qfunc} and~\eqref{eq:vi_vfunc}, until the intended $h$-stage-to-go value function is computed.

Despite the ease with which VI can be modified to solve PHMDPs, two key challenges still remain: (i) dealing with infinitely many states in {\footnotesize $\vec{x}$} and {\footnotesize $\vec{\theta}$}; and (ii) determining restrictions on {\footnotesize $Q^{h}(\vec{d}, \vec{x}, a; \vec{\theta})$} and {\footnotesize $V^{h}(\vec{d}, \vec{x}; \vec{\theta})$} that guarantee tractable solutions. In the next section we show that a parameterized extension of symbolic dynamic programming can be used to address these concerns and solve PHMDPs, optimally and in closed-form, for arbitrary horizons.