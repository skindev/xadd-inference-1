\section{Related Work}
\label{sec:background}

%\begin{itemize}
%    \item MOMDPs - Pareto front methods (approximate), convex hull algorithm (exact) -- works for discrete parameterized MDPs, but not for factored or hybrid case -- we want to look at parameterized hybrid, no exact work in this space
%    \item Sensitivity analysis of MDPs -- but none exact in hybrid case
%    \item Policy gradient -- everything is numerically oriented, never a closed-form exact solution as a function of parameter, e.g., to analyze stability
%\end{itemize}

In this work, we present Parameterized Hybrid MDPs as a unifying framework, which allows for multi-objective reasoning, exact sensitivity analysis and nonlinear parameterized policy optimization. While there are no comparable frameworks that allow for the same breadth of functionality, we briefly survey prior art in each of these three areas and conclude with a discussion of alternate uses of the term \emph{parameterized} in the MDP literature that contrasts with our contributions in this work.
%to establish the fundamental importance of each of these problems.

The Multi-objective MDP (MOMDP) literature studies the solution and analysis of trade-offs in the presence of multiple reward criteria. The techniques used to solve MOMDPs with unknown preferences depend on the nature of the scalarization function used to weight each reward component~\parencite{Roijers_JAIR_2013}. If the scalarization function is linear, then methods such as the Convex Hull Value Iteration algorithm~\parencite{Barrett_ICML_2008}, which returns the optimal policy for discrete \emph{enumerated state} MOMDPs with any linear preference function, can be used. For nonlinear scalarization functions the Pareto front of a set of undominated policies must be returned. The Pareto front can be prohibitively large and as a result solution techniques such as those of~\parencite{Chatterjee_STACS_2006} and~\parencite{Pirotta_AAAI_2015} have focused on approximating the Pareto front, or using a refinement of Pareto dominance known as Lorenz optimality~\parencite{Perny_AAAI_2013}, to further restrict the size of the solution set.  %User-oriented tools like (Interactive) Decision Maps~\parencite{Lotov_IDM} allow one to visualize and analyze the reward trade-offs associated with all parameter settings.  
In this work we present \textit{exact} \emph{factored hybrid} MOMDP solutions via the framework of PHMDPs and SDP.
%multi-objective closed-form solutions to the more general class of
%\textit{parameterized hybrid} MDPs with factored states.

Sensitivity analysis of MDP parameters is a critical tool in understanding what parameters are most important to measure well and how MDP policies vary over a range of parameters.  To date, most work has focused on uncertainty within the specification of the transition function~\parencite{Kalyanasundaram_AJC_2004}, reward function~\parencite{Tan_JAP_2011, Hopp_JOTA_1988}, or a combination of both~\parencite{Givan_AI_2000}, in discrete MDPs. The framework that we introduce in this paper enables \textit{exact} sensitivity analysis for PHMDPs that allows it to be applied in continuous state settings and permits the derivation and analysis of the \emph{optimal} policy as a function of these parameters.

Policy gradient methods rely upon optimizing parameterized policies with respect to the expected return by gradient descent. The main
concern of such methods is obtaining a good estimator of the policy gradient~\parencite{Peters_IRS_2006}. Two of the most prominent
approaches have been the finite-difference methods such as those of~\parencite{Ng_UAI_2000} and Monte Carlo methods such as~\parencite{Sutton_NIPS_1999,Baxter_ISCAS_2000}, both of which are numerically oriented and sample based. Our use of PHMDPs and SDP allows for us to solve for an \emph{exact} policy value as a parameterized function of policy parameters, hence permitting exact gradient analysis, as well as the direct application of (non-)convex optimization to this parametric value form.
% Have to cite Sutton!  If anything, drop Baxter but try not to.
% Also ``non-convex functions of parameters'', not ``non-convex parameters''

Finally, as a point of differentiation from other uses of the term \emph{parameterized} in the MDP literature, we remark that other works~\parencite{Duff_UMA_2002,Dearden_UAI_1999,Gopalan_COLT_2015} have used Parameterized MDP to refer to MDPs with latent parameters whose beliefs can be updated by observing reward and transition samples. In contrast, in this work we assume strict uncertainty of continuous MDP parameters in models that are otherwise fully specified; in this way we can treat parameters simply as free variables that can be parametrically analyzed via recent advances in symbolic solution methods and non-convex optimizers~\parencite{Gao2013}.


