\section{Introduction}
\label{sec:introduction}

Markov Decision Processes (MDPs) are the de facto standard framework for decision theoretic planning in fully observable environments~\cite{Boutilier_JAIR_1999}. Traditional MDP solution techniques often assume that the parameters of the model are known. However, in practice, model parameters are usually estimated from limited data or elicited from humans and are naturally uncertain. Hence decision analysis w.r.t. unknown parameters is a critical task in decision-making under uncertainty with applications to: (i) perform inverse learning of parameters of multi-objective rewards; (ii) perform sensitivity analyses of policies to various parameter settings; and (iii) analyze and optimize policy performance as a function of policy parameters. Formalizing models to address each of the aforementioned use cases is often fraught, due to the specification leading to hybrid (mixed discrete and continuous state and/or action) MDPs with nonlinear and/or piecewise structure that have been traditionally very difficult to solve.

In this paper we make the following key contributions:
\begin{itemize}
\item We present {\it Parameterized Hybrid MDPs} (PHMDPs) as a unified model of the aforementioned use cases and provide an algorithm that solves PHMDPs exactly and in closed-form by defining a parameterized variant of Symbolic Dynamic Programming (SDP)~\cite{Boutilier_IJCAI_2001} extended to hybrid MDPs~\cite{Sanner_UAI_2011}. 
%\item We use the PHMDP framework in conjunction with parameterized SDP to 
\item We provide the \textit{first} completely symbolic encodings of the aforementioned use cases, which in turn enables the use of recent advances in symbolic non-convex optimization techniques with \textit{guarantees}~\cite{Gao2013}.
\item We present the \textit{first exact} symbolic analysis of vaccination policies in an SIR epidemiological model~\cite{KermackMcKendrick_1927}, as well exact solutions to the inverse learning of parameters in a multi-objective reward domain and sensitivity analyses of portfolio execution strategies.
\end{itemize}