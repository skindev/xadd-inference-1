\section{Results}
\label{sec:results}

In this section we demonstrate the efficacy and tractability of PHMDps by calculating the first known optimal solutions to three difficult nonlinear sequential decision problems. We note that while dOp~\cite{Gao2013} offers strong {\footnotesize $ \delta $}-optimality guarantees, we found that nonlinear solvers such as $ \mathtt{fmincon} $~\cite{MATLAB_2010}, an interior-point algorithm, perform comparably well at optimization and are much more efficient, hence we use $ \mathtt{fmincon} $.

{\centering
    \begin{figure}[ht]
        \begin{tabular}{cc}
            \begin{subfigure}{0.24\textwidth}\centering\includegraphics[width=\textwidth]{images/robot_vf_new}\caption{{\footnotesize $V^{\pi^{*}}(loc, w_2; w_1 = 1.0)$} }\label{fig:navigation_vf}\end{subfigure}&
            \begin{subfigure}{0.24\textwidth}\centering\includegraphics[width=\textwidth]{images/sir_vf_new}\caption{{\footnotesize $V^{\pi^{*}}(\beta, \nu; s, i, r, \lambda, \\\hspace{\textwidth}\qquad cost_{\mathtt{inf}}, cost_{\mathtt{vaccine}})$}}\label{fig:sir_vf}\end{subfigure}
            \\
            \begin{subfigure}{0.24\textwidth}\centering\includegraphics[width=\textwidth]{images/oe_vf_new}\caption{{\footnotesize $V^{\pi^{*}}(\theta, inv; \\\hspace{\textwidth}\qquad p = 55.0, \kappa = 0.165)$}}\label{fig:oe_vf}\end{subfigure}&
            \begin{subfigure}{0.24\textwidth}\centering\includegraphics[width=\textwidth]{images/oe_vf_deriv_new}\caption{{\footnotesize $\nabla_{\theta} V^{\pi^{*}} (\theta, inv; \\\hspace{\textwidth}\qquad p = 55.0, \kappa = 0.165)    $}}\label{fig:oe_vf_deriv}\end{subfigure}\\            
        \end{tabular}
        \caption{Optimal Value functions for each domain.}        
        \label{tab:vf_Results}
        \vspace{-3mm}
    \end{figure}
}

{\centering
    \begin{figure}[ht]
        \begin{tabular}{cc}
            \begin{subfigure}{0.2\textwidth}\centering\includegraphics[width=\textwidth]{images/robot_opt_new}\caption{Max {\footnotesize $w_2 \in \left[0.0, 50.0 \right]$} for $ \tilde{\pi} $}\label{fig:navigation_opt}\end{subfigure}&            
            \begin{subfigure}{0.2\textwidth}\centering\includegraphics[width=\textwidth]{images/sir_opt_new}\caption{Optimal {\footnotesize $ \nu $} for {\footnotesize $ \beta \in \left[ 0.0, 1.0 \right] $}}\label{fig:sir_opt}\end{subfigure}
            \\
%            \begin{subfigure}{0.2\textwidth}\centering\includegraphics[width=\textwidth]{images/oe_opt_new}\caption{Optimal {\footnotesize $ \theta $} for {\footnotesize $ inv \in \left(0.0, 1000.0 \right) $}}\label{fig:oe_opt}\end{subfigure}&
%            \\
        \end{tabular}
        \caption{Nonlinear optimization for Navigation and SIR.}        
        \label{tab:opt_results}
        \vspace{-3mm}
    \end{figure}
}

\subsection{Inverse Learning for Navigation}
\label{sec:results_navigation}

%In this domain we consider an autonomous vehicle moving in one dimension towards a goal region. At each stage the vehicle faces a trade-off between reaching the goal region and incurring a movement cost. 
The domain is specified as follows: {\footnotesize $ \State = \left\langle loc \right\rangle$}, where $ loc $ is the location of the vehicle. {\footnotesize $ \Action \in \left\lbrace 0.0, 5.0 \right\rbrace $} is the amount by which vehicle moves relative to its current location. {\footnotesize $ \Transition\left( loc' | loc, a \right) = \delta \left[ loc' + (loc + a) \right] $}, where {\footnotesize $ a \in \Action $}. {\footnotesize $ \Reward\left(\vec{w}, loc, loc'\right) = w_1 \cdot \Reward_{\mathtt{region}} + w_2 \cdot \Reward_{\mathtt{move}} $} where,

{\footnotesize 
    \abovedisplayskip=10pt
    \belowdisplayskip=0pt
    \renewcommand{\arraystretch}{1.5}
    \begin{tabular}{ll}    
        $ \Reward_{\mathtt{region}}(loc') = $ &  $ \Reward_{\mathtt{move}}(loc, loc') =  $ \\
        \qquad $ \begin{cases}
        (loc' \leq 10.0 ) : 				& loc' \\
        \text{otherwise} : 					& 0.0 \\
        \end{cases} $ 						& \qquad $ - (loc' - loc)  $\\
%        $ \Reward_{\mathtt{move}}(loc, loc') = - (loc' - loc)$ & $ $ \\                        
    \end{tabular}
} 

%The 2a explanation does not explain the two different slopes, which is key to explain. 


Figure~\ref{fig:navigation_vf} shows the optimal value function at {\footnotesize$ \Horizon = 15 $} and reveals the trade-off between reaching the goal region and incurring a movement cost. The vehicle will incur the movement cost as long as it is mitigated by the reward gained within the goal region. Furthermore, the range of acceptable non-zero movement costs decreases the further the vehicle is from the goal region. 
%For example, when {\footnotesize $loc = -20.0$} the vehicle will only move if the cost is close to {\footnotesize $0.0$}, whereas when the {\footnotesize $loc = 0.0$}, the vehicle is willing to incur higher movement costs. 
In Figure~\ref{fig:navigation_opt} we utilise Equation~\eqref{eq:irl_q} to learn the parameters (weights) of the multi-objective reward under the following sub-optimal policy: {\footnotesize $ \tilde{\pi}(0 < loc < 10) = 5.0,  \tilde{\pi}(loc < 0 \,\mathrm{or}\, loc > 10) = 0.0$}. We observe that when {\footnotesize $a = 0.0$}, {\footnotesize $ w_2 $} is at its maximum allowable value. When {\footnotesize $a = 5.0$}, in the region {\footnotesize $ (5 < loc < 10) $}, the rate at which {\footnotesize $ w_2 $} increased is dependent on the distance of the vehicle from the goal region and the movement size.

% step size is 5.0
% when 0 < loc < 5 it is linear with flatter gradient
% when 5 < loc < 10 < it is linear with steeper gradient

\subsection{SIR Epidemic}
\label{sec:results_influenza}

%In this domain we address the optimization of a static parameterized vaccination policy under a model of Influenza epidemiology where the decision maker must balance the cost of vaccination and the burden of disease.
% on the population. 
The well studied SIR epidemic~\cite{KermackMcKendrick_1927} domain is specified as follows: {\footnotesize $ \State = \left\langle s, i, r \right\rangle$}, where $ s $, $ i $, and $ r $ refer to the size of the susceptible, infected and recovered sub-populations, respectively. {\footnotesize $ \Action \in \left\lbrace \pi(\nu) \right\rbrace $} where {\footnotesize $\nu \in \left[0.0, 1.0\right]$} is the proportion of $ s $ to vaccinate at each stage. The transition function {\footnotesize \Transition} for each state variable in {\footnotesize \State} is given by:
    {\footnotesize 
        \abovedisplayskip=5pt
        \belowdisplayskip=0pt
        \renewcommand{\arraystretch}{1.5}
        \begin{tabular}{ll}
            $ \Transition\left( s' | s, i, r, \pi(\nu) \right) =$ & $ \delta \left[ s' - (s - \beta \cdot s \cdot i - \pi(\nu) \cdot s) \right] $ \\
            $ \Transition\left( i' | s, i, r, \pi(\nu) \right) =$ & $ \delta \left[ i' - (i + \beta \cdot s \cdot i - \lambda \cdot i) \right] $ \\
            $ \Transition\left( r' | s, i, r, \pi(\nu) \right) =$ & $ \delta \left[ r' - (r + \lambda \cdot i + \pi(\nu) \cdot s) \right] $ \\            
        \end{tabular}
    }%
where {\footnotesize $ \beta $} is the infection rate and {\footnotesize $\lambda$} is the spontaneous recovery rate. The reward is specified as {\footnotesize $ \Reward\left(cost_{\mathtt{inf}}, cost_{\mathtt{vaccine}}, s, i, r, \pi(\nu) \right) = (s \cdot (-cost_{\mathtt{vaccine}} \cdot \pi(\nu) + (1 - \pi(\nu)))) - cost_{\mathtt{inf}} \cdot i + r$}. {\footnotesize $ cost_{\mathtt{inf}} $} is the incident cost of infection and {\footnotesize $ cost_{\mathtt{vaccine}} $} is the unit cost of vaccination. We assume that the total population is constant and that vaccinated individuals go straight from {\footnotesize $ s $} to {\footnotesize $ r $} without being infected. The decision maker must balance the cost of vaccination and the burden of disease on the population. 

Figure~\ref{fig:sir_vf} shows the optimal value function at {\footnotesize$ \Horizon = 7 $} when {\footnotesize $ s = 1000.0, i = 100.0, r = 0.0, \lambda = 0.25 $}, {\footnotesize $ cost_{\mathtt{vaccine}} = 4.0$} and {\footnotesize $ cost_{\mathtt{inf}} = 10.0 $}. The value function shows that it is not always optimal to vaccinate the entire population. In fact, Figure~\ref{fig:sir_opt} reveals that vaccinating the entire population is only optimal when {\footnotesize $ \beta > 0.25 $}, that is, when the \textit{basic reproductive ratio} {\footnotesize $ R_0 \,(= \beta/\lambda)$}~\cite{Heffernan_2005} exceeds 1.0. Scenarios where {\footnotesize $R_0 > 1.0$} can lead to an epidemic. 

%We remark that, as far as we are aware, 
To the best of our knowledge, this is the \textit{first} exact symbolic analysis of vaccination policies in an SIR epidemiological model. Furthermore, PHMDPs and SDP can be used to solve \textit{any} SIR model without needing an analytical solution.
%, such as those specified in~\cite{LekoneFinkenstaedt_2006,CoburnWagnerBlower_2009}, without needing an analytical solution. 

\subsection{Optimal Execution}
\label{sec:results_oe}

%In this domain we examine the sensitivity of an optimal portfolio transaction model to its parameters. Institutional investors often want to transact a number of shares that exceeds available liquidity. In these situations they face a clear trade-off between the market impact of transacting immediately and the volatility of slow execution. 
The domain is specified as follows {\footnotesize $ \State = \left\langle p, inv \right\rangle$}, where $ p $ is the price of the asset and $ inv $ is the inventory remaining. {\footnotesize $ \Action \in \left\lbrace \pi\left( \theta \right) \right\rbrace$}, where {\footnotesize $ \theta \in \left( 0.0, 1.0\right)$} is the proportion of inventory to be sold. The transition function {\footnotesize \Transition} for each state variable in {\footnotesize \State} is given by:
{\footnotesize 
    \abovedisplayskip=5pt
    \belowdisplayskip=0pt
    \renewcommand{\arraystretch}{1.5}
    \begin{tabular}{ll}
        $\Transition\left( p' | p, inv, \pi\left( \theta \right) \right) =$ & $ \delta \left[ p' - (p - \kappa \cdot (inv \cdot \pi\left( \theta \right)) + \epsilon) \right] $ \\
        $\Transition\left( inv' | p, inv, \pi\left( \theta \right) \right) =$ & $\delta \left[ inv' - (inv - inv \cdot \pi\left( \theta \right)) \right] $ \\
    \end{tabular}
}%
where {\footnotesize $ \kappa > 0$} is a market-impact parameter and {\footnotesize $ \epsilon $} is a discrete noise parameter. The reward is specified by {\footnotesize $ \Reward\left(p', inv, \pi\left( \theta \right) \right) = p' \cdot inv \cdot \pi\left( \theta \right)$ }. Institutional investors often face a clear trade-off between the market impact of transacting a large number of shares immediately and the volatility of slow execution. 

Figures~\ref{fig:oe_vf} and~\ref{fig:oe_vf_deriv} show the optimal value function at {\footnotesize $ \Horizon = 10 $} and its derivative with respect to the parameter {\footnotesize $ \theta$}, respectively. When inventory is low, the value function is high at higher {\footnotesize $ \theta$} and the corresponding derivative is relatively stable. When the inventory is high, the value function is high at lower {\footnotesize $ \theta$} and the corresponding derivative shows maximum sensitivity. This indicates that when inventory is low, selling a large proportion of shares allows the investor to capture the current price and when inventory is high, selling a lower proportion of shares captures a more stable set of future prices. 

\subsection{Time and Space Complexity}

%{\centering
    \begin{figure}[ht]
        \begin{tabular}{cc}
            \begin{subfigure}[b]{0.24\textwidth}\centering \includegraphics[width=\textwidth]{images/time_plot_new}
                \label{fig:time_complexity}\end{subfigure}\hspace{-1em}
            \begin{subfigure}[b]{0.24\textwidth}\centering \includegraphics[width=\textwidth]{images/space_plot_new}
                \label{fig:space_complexity}\end{subfigure}
            \\
        \end{tabular}
        \vspace{-2em}
        \caption{Time and Space versus {\footnotesize $ \Horizon $} for Navigation.}
        \label{fig:time_space_complexity}    
%        \vspace{-3mm}
    \end{figure}
%}

Figure~\ref{fig:time_space_complexity} shows an approximate linear relationship between the horizon {\footnotesize $ \Horizon $} and the computational time and space for the navigation domain, which is a promising scalability property of the overall framework.